This file contains code related preld, a program that predicts $\sigma_d^2$ from assumptions about population history.

\label{preld_preld}%
\hypertarget{preld_preld}{}%
\subsection*{{\ttfamily preld}, a program that predicts $\sigma_d^2$ from population history }

Parameter values, including those describing population history, may be read either from the initialization file {\ttfamily ldpsiz.\+ini} or specified on the command line. Command-\/line arguments override values in the initialization file. Several methods are implemented for predicting L\+D, and one or more of these may be specified either in the initialization file or via the {\ttfamily -\/-\/methods} argument. For the current list of available methods, see \hyperlink{model_8h_a1c02862e00fd2fd232c5ba777beadc84}{Model\+\_\+alloc}.

\subsubsection*{Usage }

\begin{DoxyVerb}usage: preld [options]
   where options may include:
   -u \<x\> or --mutation \<x\>
      set mutation rate/generation
   -b \<x\> or --nbins \<x\>
      specify the number of recombination rates
   -r \<x\> or --lo_r \<x\>
      low end of range of recombination rates in centimorgans
   -R \<x\> or --hi_r \<x\>
      high end of range of recombination rates in centimorgans
   -e or --equilibria
      show equilibrium for each epoch
   --log
      print log10 of sigdsq values
   -n or --twoNsmp
      set haploid sample size
   -E or --nextepoch
      move to next earlier epoch
   --twoN \<x\>
      set haploid pop size to x in current epoch
   -T \<x\> or --time \<x\>
      set length of current epoch to x generations
   -S or --printState
      print state vectors
   -m <method list> or --methods <method list>
      specify methods
   --exact
      don't use ODE to approximate difference equations
   -h or --help
      print this message
\end{DoxyVerb}


\begin{DoxyCopyright}{Copyright}
Copyright (c) 2014, Alan R. Rogers \href{mailto:rogers@anthro.utah.edu}{\tt rogers@anthro.\+utah.\+edu}. This file is released under the Internet Systems Consortium License, which can be found in file \char`\"{}\+L\+I\+C\+E\+N\+S\+E\char`\"{}. 
\end{DoxyCopyright}
